\chapter{Factor Graphs and Bayes Tree}

In this chapter, I will introduce several key basic concepts and the associated terminologies of \textit{factor graph} \cite{factorgraph} along with its SLAM formulation as proposed by Kaess and Dellart [ref] but for the scenario of multiple robots. A factor graph is a type of probabilistic graphical model which represents the factorization of a probabilistic distribution function. They are used to model complex estimation problems having wide range of applications in robotics. Formulating the SLAM problem using the factor graph, also known as pose graph in the robotics literature, opens the door for the application of several probabilistic inference algorithms. Our research is multi-robot SLAM and we will use this as an example throughout the paper. In the case of SLAM, the set of constraints obtained from the proprioceptive sensors like odometry measurements, inertial measurement units (IMUs) and exteroceptive sensors like range (LIDAR) and vision measurements form a Markov chain that connects all the variables to be estimated. Also, the solution to the SLAM problem using the recent pose graph representations has garnered much attention because of their computational efficiency and robustness. 





